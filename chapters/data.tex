% !TeX root = ../main.tex

\chapter{实验数据}

\section{数据来源}

\subsection{探测器数据}
分析所用的数据集是在2015年至2018年期间,由ATLAS探测器通过LHC上质子--质子(pp)碰撞($\sqrt{s} = 13\,\mathrm{TeV}$)收集的。
该数据集的总积分亮度为 $140\,\mathrm{fb}^{-1}$。排除了LHC束流不稳定或并非所有子探测器都在运行的数据~\cite{ATLAS_data_quality}。

ATLAS 探测器使用分层触发系统记录事件。
一级 (L1) 触发器采用定制电子设备实现,可将事件发生率从 LHC 交叉频率 40 MHz 降低到设计值 100 kHz。
第二级触发器称为高级触发器 (HLT),由运行在商用计算机集群上的软件实现,该软件可处理事件并将记录事件的发生率降低到 1 kHz。
软件套件用于数据模拟、真实数据和模拟数据的重建和分析、探测器操作以及实验的触发和数据采集系统。

对于 CalR+2J 通道,使用了专门的LLP触发器(称为“CalRatio触发器”)。
在一级触发阶段,喷注在 $\eta \times \phi$ 平面中的 $0.2 \times 0.2$ 区域进行重建,这是基于信号喷注通常比标准模型喷注更窄的事实。
选取的事件要求喷注的横能量 $E_T$ 超过 60 GeV 或 100 GeV(具体取决于数据收集时期)。
在某些运行阶段,存在一些触发器,其要求仅在喷注所在的 $\eta - \phi$ 位置上电磁量能器中没有能量沉积时才接受L1喷注对象。
该要求可作为 CalRatio 指标的替代,使得 $E_T$ 阈值可降低至 30\,GeV。

在高级触发阶段,触发喷注还需满足 $|\eta| < 2.5$ 的条件,即在可获得径迹信息的范围内。
此外,该喷注还应具有高比例的强子量能器能量沉积,并满足修改后的噪声抑制条件,
并通过一种用于去除束流背景(Beam Induced Background,BIB)的算法(例如利用能量沉积在 $\phi$ 方向上的时间和对准信息)。
在下文中,使用 CalRatio 触发器所收集的数据集称为主数据集。

在其余通道中,使用了单轻子或双轻子触发器,其中仅考虑在某一年中整个数据收集期间未被预缩(unprescaled)运行的触发器。

为研究非碰撞背景(NCB)以及构建验证区和控制区,额外收集了三个数据集:
BIB 数据集包含满足 CalRatio 触发器所有要求但未通过 BIB 去除算法的事件。
宇宙射线数据集使用相同的触发器选择标准,但来自空束团交叉期间记录的事件。
双喷注(dijet)数据集是通过基于单喷注触发器~\cite{trigger} 的触发器选择的,并排除了 CalRatio 触发器,从而确保其与主数据集正交。
该数据集用于测试位移喷注神经网络标注器(displaced jet neural network tagger)输出中建模的影响。

\subsection{模拟数据}
\subsubsection{背景数据}
主要的标准模型背景包括 CalR+2J 通道中的标准模型多喷注事件,以及 CalR+W 和 CalR+Z 通道中的 W/Z+jets、$t\bar{t}$ 和单顶夸克事件。
尽管最终的背景估计使用的是数据驱动方法,但仍需要蒙特卡洛(Monte Carlo)模拟事件来训练机器学习(Machine Learning)判别器并评估某些系统不确定性。

标准模型多喷注样本使用 \Pythia 8.186~\cite{pythia} 在领头阶(Leading Order)精度生成,
采用 A14 参数调节集~\cite{Pythia_tunes} 进行部分子簇射和强子化的计算。
使用的部分子分布函数(Parton Distribution Function)为 \NNPDF[2.3lo]~\cite{parton_PDF}。

包含 W 或 Z 玻色子与喷注的事件使用 \Sherpa[v2.2.1]~\cite{Sherpa} 进行模拟,
最多两个喷注的矩阵元计算精确到次领头阶(Next-to-Leading Order),
最多四个喷注的为 LO 精度,采用 \textsc{Comix}~\cite{Comix} 和 \OPENLOOPS~\cite{Open_Loops} 库计算,
部分子簇射使用 \Sherpa 默认设置~\cite{parton_shower}。

$t\bar{t}$ 事件的产生使用 \POWHEGBOX[v2]~\cite{POWHEG_BOX} 在 NLO 精度下模拟,
使用 \NNPDF[3.0NLO] PDF 集,并将 hdamp 参数设为 1.5 倍的顶夸克质量~\cite{ATL-PHYS-PUB-2016-020}。
事件通过 \Pythia[8.230] 与 A14 参数调节集 和 \NNPDF[2.3LO] PDF 集进行部分子簇射和强子化。
$t\bar{t}$ 包含产生的 NLO 截面被修正为 QCD 的 NNLO 理论预测,并包括通过 Top++2.0~\cite{Beneke:2011mq} 所计算的 NNLL 软胶子重求和。

单顶夸克的产生使用 \POWHEGBOX[v2] 在五味方案下的 NLO QCD 精度模拟,使用 NNPDF3.0NLO PDF 集。
为处理与 $t\bar{t}$ 产生的干涉,采用了图移除(diagram-removal)方案~\cite{Frixione:2008yi}。
事件也使用 \Pythia{8.230} 进行簇射与强子化。
其总截面被修正为带有 NNLL 软胶子修正的 NLO QCD 理论预测~\cite{Aliev:2010zk}。

\subsubsection{信号数据}
研究中考虑了三种基准信号模型。第一种是隐藏区域(HS)模型,其中标量玻色子 $\Phi$(可以是希格斯玻色子,
也可以是其他行为类似的较轻或较重粒子)作为标准模型与隐藏区域之间的媒介。
$\Phi$ 可衰变为中性的长寿命标量粒子 $s$,通常衰变为运动学允许的最重费米子对,通常为 $b$ 夸克对。
事件使用 \MADGRAPH{5\_aMC@NLO v2.6.2}~\cite{Alwall:2014hca} 在 LO 精度下生成,使用 NNPDF2.3LO PDF 集。
考虑 $\Phi$ 的两种产生模式:CalR+2J 通道中为胶子-胶子融合(ggF),CalR+W/Z 通道中为伴随矢量玻色子(W 或 Z)产生。
不考虑矢量玻色子融合(VBF)模式:在 CalR+2J 通道中,所有选择对象均来源于 LLP 的衰变,因此 VBF 与 ggF 的产率预期相似。
由于两种产生模式的相对截面,没有预期来自 VBF 模式的显著附加贡献。
伴随 W 或 Z 玻色子产生的 HS 样本分别称为 WHS 和 ZHS。
对于胶子融合样本,对 $\Phi$ 的横动量分布进行重加权,使其与同一生成器的 NLO 预测相匹配。
归一化时假设 $\Phi$ 为 标准模型 希格斯玻色子时,其产生截面为:胶子融合 $48.6$\,pb,
伴随 W 玻色子(轻子衰变)$0.45$\,pb,伴随 Z 玻色子(轻子衰变)$0.09$\,pb(均来自 NNLO 计算)。
生成了多个样本集合,粒子质量假设范围为:介子 $60$--$1000$\,GeV,长寿命标量 $5$--$475$\,GeV。

第二种信号模型应用于 CalR+W/Z 通道,是辐射自矢量玻色子并衰变为胶子的厌光轴轻子(photo-phobic ALPs)~\cite{Brivio:2017ije}。
ALP 质量范围为 $0.1$--$40$\,GeV,其与胶子的耦合决定其寿命,范围为 $10^{-7}$ 到 $10^{-2}$。
此模型中仅考虑矢量玻色子衰变为电子或 muon 的情形。
ALP 样本使用 \MADGRAPH{5\_aMC@NLO v2.9.3} 在 LO 精度下生成,使用 NNPDF2.3LO PDF 集,分别称为 ZALP 和 WALP。

第三种基准模型为长寿命暗光子($Z_d$)模型~\cite{Davoudiasl:2012ag, Davoudiasl:2013aya},
其中 $Z_d$ 是与 Z 一同在标量介子衰变中产生的,称为 HZZd 模型。
考虑的介子质量范围为 $250$--$600$\,GeV,暗光子质量范围为 $5$--$400$\,GeV。
事件使用  \MADGRAPH{5\_aMC@NLO v2.6.7} 在 LO 精度下生成,PDF 为 NNPDF2.3LO。
所有信号样本的部分子簇射和强子化均使用 \Pythia{8} 并采用 A14 tune。


\section{数据处理}

