% !TeX root = ../main.tex

\chapter{简介}

2012年由大型强子对撞机(LHC)上的ATLAS与CMS实验发现的Higgs粒子补全了标准模型的最后一片拼图,标准模型因其精准预言也被誉为史上最成功的物理模型。
但标准模型并不能解释目前发现的所有实验现象,例如暗物质与暗能量、中微子质量来源,并存在精细微调(fine-tuning)问题,因而寻找超越标准模型的新粒子成为大型强子对撞机目前的首要任务。

本课题研究的物理过程为Hidden Sector暗物质模型中的Higgs-like中介粒子衰变为一对长寿命的中性标量粒子,该中性标量粒子继续衰变为标准模型粒子。
该过程会在ATLAS探测器内留下的信号为偏离对撞顶点(displaced vertices)的径迹,或是在电磁量能器沉积少量能量的无径迹事例,又或是无径迹也无量能器能量沉积的类$\mu$子事例。

为了鉴别数据中可能存在的信号,对每个喷注(jet)使用神经网络进行信号与本底的分类。ATLAS实验现有的神经网络分为主网络与对抗网络两部分,其中主网络用于分类、对抗网络用于减小数据与蒙特卡洛模拟之间的差别。主网络的结构为一维卷积网络加上长短期记忆网络(LSTM),对抗网络为前向网络。

本课题预期通过改变网络架构或调整训练策略以提升神经网络的性能,进而减小神经网络为分析带来的系统误差、提升寻找长寿命粒子的灵敏度。
