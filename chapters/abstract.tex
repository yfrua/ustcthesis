% !TeX root = ../main.tex

\ustcsetup{
  keywords  = {质子对撞, 长寿命粒子, 量能器, 图神经网络},
  keywords* = {Proton-Proton Scattering, Long‐Lived Particle, Calorimeter, Graphic Neural Network},
}

\begin{abstract}
  本论文基于 ATLAS 探测器量能器信息,提出并实现了一种多阶段神经网络分类架构,用以识别由长寿命粒子衰变产生的位移喷注。
  选用的数据通过 CalRatio 触发与喷注预选,以选取具有显著强子量能器能量沉积比例且喷注宽度窄的候选事件。
  引入图神经网络对喷注产生的径迹、量能器、 μ 子谱仪信息进行深度特征提取,实现了对信号与背景的有效区分。
  同时网络训练过程中引入控制区喷注信息,利用该补充信息减小了真实数据与模拟模拟间的差异。
  该方法相较于传统一维卷积加上 LSTM 网络,将背景区分能力提升了一倍,同时有效减小了在真实数据与模拟数据之间的错误建模。

\end{abstract}

\begin{abstract*}
  This thesis proposes and implements a multi-stage neural network classifier based on calorimeter information from the ATLAS detector,
  aiming to identify displaced jets originating from the decays of long-lived particles.
  Candidate events were selected using the CalRatio trigger and jet preselection,
  targeting those with a large hadronic calorimeter deposited energy fraction and narrow jet width.
  A graphic neural network was introduced to perform deep feature extraction from track, calorimeter,
  and muon spectrometer information associated with the jets, enabling effective discrimination between signal and background.
  During network training, jet information from control regions was incorporated
  to reduce discrepancies between real and simulated data using this complementary input.
  Compared with a traditional architecture combining one-dimensional convolution and LSTM networks,
  this method doubled the background rejection and significantly mitigated the mismodeling between data and simulation.
\end{abstract*}
