% !TeX root = ../main.tex

\chapter{BDT 的输入变量}
\label{cpm:BDT}

以下是用于逐事件 BDT 训练的输入变量列表:

\begin{itemize}
    \item 每个 CalRatio 候选信号喷注($\text{jet}^{\text{sig}1}$、$\text{jet}^{\text{sig}2}$)
          与 BIB 候选喷注($\text{jet}^{\text{BIB}1}$、$\text{jet}^{\text{BIB}2}$)
          的喷注 NN 信号得分与 BIB 得分,

    \item CalRatio 喷注候选($\text{jet}^{\text{sig}1}$、$\text{jet}^{\text{sig}2}$)的 $p_T$;

    \item $H_T^{\text{miss}} / H_T$,其中 $H_T$ 是所有 $p_T > 30$\,GeV 且 $|\eta| < 3.2$ 的喷注的 $p_T$ 的标量和,
          $H_T^{\text{miss}}$ 是这些喷注 $p_T$ 的矢量和的负值的模长;

    \item $M_{\text{eff}} = H_T + H_T^{\text{miss}} $;

    \item $\Delta \phi (\text{jet}^{\text{sig}1}, \text{jet}^{\text{sig}2})$ 和
          $\Delta R(\text{jet}^{\text{sig}1}, \text{jet}^{\text{sig}2})$,
          即事件中两个候选信号喷注之间的角距离;

    \item 所有事件中 clean 喷注的信号 NN 权重的均值。
          这些变量与列表中第一项所述的每喷注信号 NN 权重相关,但它们在以下情况下提供了额外的判别能力:
          当一个(或两个) LLP 衰变被重建为两个分辨的喷注时,可能会出现第三个(甚至第四个)具有较高信号 NN 权重的喷注。
          这一点在高质量样本中尤其重要,因为这些附加喷注的 $p_T$ 足够高,可以被归类为 clean 喷注。
          而在 BIB 样本中出现此类情形的概率则要低得多。均值的计算方式利用了第三和第四个类信号喷注的信息,而无需对每个事件中 clean 喷注的数量作出特定要求。

    \item 所有事件中 clean 喷注的 BIB NN 权重的均值。
          上一项中关于使用第三和第四个类信号喷注的讨论,同样适用于这些变量。
          在这种情况下,BIB 样本比任何一个信号样本更有可能出现第三或第四个具有较高 BIB NN 权重的 clean 喷注。
\end{itemize}
