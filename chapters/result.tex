% !TeX root = ../main.tex

\chapter{结论}
本工作基于 ATLAS 探测器量能器信息,提出并实现了一种多阶段神经网络分类架构,用以识别由长寿命粒子衰变产生的位移喷注。首先,通过 CalRatio 触发与喷注预选,选取了具有显著强子能量沉积且径迹稀疏的候选事件。随后,引入图神经网络对喷注内部拓扑、径迹及 μ子系统信息进行深度特征提取,并以其输出结合多种事件级物理量作为 BDT 输入,实现了对信号与背景的有效区分。该方法在真实数据验证中,相较于传统一维卷积+LSTM 网络,将信号效率提高了 25\%–35\%,显著提升了对束流背景和 pileup 的鲁棒性,并拓展了对多种 LLP 模型、不同衰变长度区间的搜索敏感度。

针对三类典型 LLP 模型(Hidden‐Sector、AXP、暗光子),本分析通过 ABCD 数据驱动方法对信号区背景进行估计,并对平均衰变长度从 10 cm 至 50 m 范围内的粒子分别给出排除限值,刷新了 ATLAS 在此领域的最优结果。研究还表明,图神经网络在处理喷注多源信息融合方面具有独特优势,可为未来 LLP 搜索提供参考。

展望未来,可进一步优化 GNN 架构并结合对抗性训练以减小模拟与数据间的差异;同时,可探索更广泛的 LLP 信号通道与探测子系统联用策略,以全面提升对新物理的探索能力。
