% !TeX root = ../main.tex

\chapter{结论}
本工作基于 ATLAS 探测器量能器信息,提出并实现了一种多阶段神经网络分类架构,用以识别由长寿命粒子衰变产生的位移喷注。
首先,通过 CalRatio 触发与喷注预选,选取了具有显著强子量能器能量沉积比例且喷注宽度窄的候选事件。
随后,引入图神经网络对喷注产生的径迹、量能器、 μ 子谱仪信息进行深度特征提取,实现了对信号与背景的有效区分。
同时网络训练过程中引入控制区喷注信息,利用该补充信息减小了真实数据与模拟模拟间的差异。
该方法相较于 Run 2 分析中的一维卷积加上 LSTM 网络,将背景区分能力提升了三倍,同时有效减小了在真实数据与模拟数据之间的错误建模。

下一步计划评估 GNN 在信号区域不同数据集的表现(根据 LLP 的质量划分),考察信号鉴别能力与系统误差。
并且尝试用 Run 3 数据集进行训练,评估 GNN 在新数据集上的表现,探究与 Run 2 数据可能存在的差别。
