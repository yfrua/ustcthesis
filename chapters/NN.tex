% !TeX root = ../main.tex

\chapter{神经网络}
\label{chap:NN}

为了识别信号中的位移喷注(displaced jets)与类标准模型(SM-like)径迹以及由束流背景(BIB)产生的伪径迹,训练了一个逐喷注(per-jet)的神经网络(NN)。
该神经网络在隐藏区域(Hidden Sector)模型信号、标准模型多喷注(SM Multi-jets)以及 BIB 的混合样本上进行训练。
神经网络的输入变量为与每个径迹相关的低级特征,涵盖整个 ATLAS 探测器的信息,包括径迹、量能器以及 μ 子谱仪相关信息。
训练前对每个变量进行了预处理,以确保神经网络训练的最优效果。
网络结构选用基于 Transformer 的图神经网络(GNN)。训练采用 Pytorch 框架,并使用 Optuna 框架进行超参数(hyperparameter)优化。
训练结果经过评估以排除过拟合现象,并在所用信号样本上相较于此前的网络性能有所提升。

训练中使用的径迹均为“clean”径迹,其定义见\autoref{sec:jet_preselection}。
训练阶段仅使用事件编号为奇数的信号事件,偶数编号的事件则用于后续分析流程,以避免因数据重叠而引发的偏差,确保最终结果的客观性和可信度。
对于 QCD 径迹,训练样本来自 JZ2W、JZ3W 和 JZ4W 样本,这些样本覆盖了 $p_{\mathrm{T}}$ 范围为 60--800 GeV 的 dijet 事件,叠加的 pileup 径迹将该范围扩展至 40 GeV。事件中 $p_{\mathrm{T}}$ 最高的两个径迹在满足公共选择标准后用于训练。BIB 径迹训练样本则取自第 5.1 节中定义的 BIB 数据集,选取触发 HLT CalRatio 触发器但未通过 HLT BIB 抑制算法的事件。随后通过 $\Delta R$ 匹配离线径迹与触发径迹,匹配成功并满足公共选择条件的径迹被标记为 BIB 径迹。

由于信号与背景径迹可通过 ATLAS 探测器的各个子系统加以刻画,因此神经网络也应当充分利用来自各子探测器的径迹信息。输入变量包括(详细变量见附录 B):

\begin{itemize}
    \item 位于 $\Delta R < 0.2$ 范围内的径迹的空间位置、动量、冲击参数和质量变量;
    \item 与径迹相关联的 topo-clusters 的动量、时间信息和空间位置;
    \item 径迹在电磁量能器与强子量能器各层的能量沉积比例;
    \item 距径迹在 $\Delta R < 0.2$ 范围内的缪子段的空间与时间信息;
    \item 径迹本身的动量与空间位置。
\end{itemize}

训练前对输入变量进行预处理以减少训练事件之间的偏差,并调整变量分布以加快训练收敛速度。预处理完成后,径迹输入至神经网络,网络输出对应于信号、QCD 或 BIB 的标签。训练程序使用的是基于 TensorFlow 后端的 Keras 框架,因其提供了对复杂网络的高度自定义能力,同时保持了快速开发的优势。

神经网络结构选用了两种模块组合:第一为一维卷积神经网络(1D CNN),作为特征提取器。卷积操作作用于每个子探测器变量形成的一维数组,分别针对跟踪器、量能器和缪子段输入,从而捕捉各输入变量间的相关性。第二部分为 LSTM 网络,其具备对连续输入之间关系的记忆能力(例如 TopoCluster 之间的联系),从而进一步利用各径迹中多个 topo-cluster、径迹段与缪子段之间的内部相关性。

各子网络的输出与径迹输入变量连接后输入至若干全连接层,最终网络输出三个预测值:signal weight、QCD weight 以及 BIB weight。网络结构参数通过超参数扫描选定。

此外,还在神经网络输出后添加了一个对抗网络(adversarial network),该网络用于在控制区中学习区分模拟径迹与真实数据径迹的特征,并将此反馈至主网络以抑制数据与 MC 的不一致性。这一对抗训练显著降低了数据与 MC 模型之间的差异,效果可见图 10 与图 11 中 BIB 分数的比较。网络结构图与对抗训练流程可见图 9。

训练过程分别针对低质量与高质量信号样本(low-mass 与 high-mass)独立进行。低质量样本中由于 $p_{\mathrm{T}}$ 较低,难以与 QCD 或 BIB 背景区分,因此测试了联合与分别训练的效果,结果表明分开训练在信号判别效率方面表现更优。因此,低质量训练使用 $m_\tau = 60, 125, 200$ GeV 样本,高质量训练使用 $m_\tau = 400, 600, 1000$ GeV 样本。

最终训练效果可见图 12--14,分数越低表明被标记为该类别径迹的可能性越低,分数接近 1 则表示为该类别径迹的置信度越高。训练约经历 50 个 epoch(遍历完整训练集)收敛。有关神经网络输入建模的不确定性的系统评估流程见附录 B.9。

\section{网络结构}

\section{训练方法}

\section{结果}

